C언어에서의 문자열은 배열이므로, 포인터 형태로 사용할 수 있다.
C언어에서의 문자열 비교, 연산, 탐색 등의 알고리즘의 사용 방법은 각각 함수 형태도 제공된다!


<문자열의 개념>
    문자들의 배열
    컴퓨터 메모리 구조상에서 마지막에 널(NULL)값을 포함한다.
    가령 HELLO WORLD라는 문자열을 배열에 저장할 때,
    실제 문자의 길이는 띄어쓰기를 포함하여 11자 이지만,
    마지막에 NULL값(\0)까지 포함하여, 12자로 저장된다.
    NULL값은 문자열을 처리할 때 굉장히 중요한 요소로 작용한다.

<문자열과 포인터>
    문자열 형태로 포인터를 사용하면 포인터에 특정한 문자열의 주소를 넣게된다.
    char *a = "Hello World";
    위와 같이 문자열 자체를 마치 상수처럼 포인터변수에 초기화해줄 수 있다.
    이러한 문자열을 "문자열 리터럴"이라고 한다.
    이는 컴파일러가 알아서 문자열 자체가 특정한 컴퓨터메모리 주소에 담길 수 있도록 남아있는 메모리 공간 중에서 주소를 결정해 준다.
    만들어진 특정한 문자열의 주소가 기록이 되고, 그 주소를 포인터 변수가 갖고 있게 된다.
    'Hello World'라는 문자열은 읽기 전용

<문자열 처리를 위한 다양한 함수>
    C언어의 문자열 처리와 관련한 기본적인 문자열 함수를 알고 있는 것이 좋다.
    나중에 C++을 이용하면 더욱 간편하고 다양한 함수를 사용할 수 있다.
    C언어에서의 문자열 함수는 <string.h>라이브러리에 포함되어 있다.

    strlen(): 문자열의 길이 반환
    strcmp(): 문자열1이 문자열2보다 사전적으로 앞에 있다면 -1, 아니면 1 반환
    strcpy(): 문자열 복사
    strcat(): 문자열1에 문자열2를 더함
    strstr(): 문자열1에 문자열2가 "어떻게 포함되어 있는지" 반환
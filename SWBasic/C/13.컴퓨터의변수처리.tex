1. C언어에서 다양한 변수를 처리하는 방법에 대하여
2. 지역변수, 전역변수, 레지스터 변수등에 대하여
3. 특정한 함수에 값을 전달하거나 주소를 전달하는 방법에 대하여

<컴퓨터가 변수를 처리하는 방법>
    <프로그램 메모리주소>
        - 컴퓨터에서 프로그램이 실행되기 위해서는 프로그램이 메모리에 적재(Load)되어야 한다.
        - 당연히 프로그램의 크기를 충당할 수 있을 만큼의 메모리 공간이 있어야 한다.
        - 일반적인 컴퓨터의 운영체제는 메모리 공간을 네가지로 구분하여 관리함
        >>  코드영역, 데이터 영역, 힙 영역, 스택 영역
            코드 영역: 소스 코드
            데이터 영역: 전역 변수, 정적 변수
            힙 영역: 동적 할당 변수
            스택 영역: 지역 변수, 매개변수(함수내부적 변수)
        
            전역 변수(Global Variable):
                - 프로그램의 어디서든 접근이 가능한 변수
                - main함수가 실행되기도 전에 프로그램의 시작과 동시에 메모리(데이터 영역)에 할당되고, "프로그램이 종료되면" 메모리에서 해제된다.
                - 프로그램의 크기가 커질수록 전역 변수로 인해 프로그램이 복잡해질 수 있다.
                - main함수 밖에 선언됨

            지역 변수(Local Variable):
                -  프로그램에서 특정한 블록(Block)에서만 접근할 수 있는 변수
                - 함수가 실행될 때마다 메모리(스택 영역)에 할당되어 "함수가 종료되면" 메모리에서 해제됨

            정적 변수(Static Varialble):
                - 특정한 블록에서만 접근할 수 있는 변수
                - 프로그램이 실행될 때 메모리(데이터 영역)에 할당되어 "프로그램이 종료되면" 메모리에서 해제된다.

            레지스터 변수(Register Variable):
                - 메인 메모리 대신 CPU의 레지스터를 사용하는 변수
                - 레지스터는 매우 한정되어 있으므로 실제로 레지스터에서 처리될지는 컴파일러가 결정하므로, 장담할 수 없음

        <함수의 매개변수가 처리될 때>
            - 함수를 호출할 떄 함수에 필요한 데이터를 매개변수로 전달.
            - 전달 방식은 1) 값에 의한 전달 방식과 2) 참조에 의한 전달 방식이 있다.
            - 값에 의한 전달 방식은 단지 값을 전달하므로 "함수 내에서 변수가 새롭게 생성"됨.
            - 참조에 의한 전달 방식은 "주소를 전달"하므로 "원래의 변수 자체에 접근"할 수 있음

            - 참조에 의한 전달 방식은 단지 매개변수로 "포인터(Pointer)" 변수를 보낼 뿐 특별한 것이 아니다.
            